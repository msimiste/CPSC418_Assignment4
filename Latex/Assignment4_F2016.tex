\documentclass{assignment}

\coursetitle{Introduction to Cryptography}
\courselabel{CPSC 418}
\exercisesheet{Home Work \#4}{}
\student{Mike Simister - 10095107}
\semester{Fall 2016}
%\usepackage[pdftex]{graphicx}
%\usepackage{subfigure}
\usepackage{amsfonts}
\usepackage[fleqn]{amsmath}
\usepackage[normalem]{ulem}
\usepackage{amsthm}
\usepackage{physics}
\usepackage[singlespacing]{setspace}
\usepackage[normalem]{ulem}

\begin{document}
\sloppy

\begin{center}
\renewcommand{\arraystretch}{2}
\begin{tabular}{|c|c|c|} \hline
Problem & Marks \\ \hline \hline
1 & \\ \hline
2 & \\ \hline
3 & \\ \hline
4 & \\ \hline
5 & \\ \hline
6 & \\ \hline
7 & \\ \hline \hline
Total & \\ \hline
\end{tabular}
\end{center}
\newpage

\begin{flushleft}
\begin{problemlist}
\pbitem (Security of RSA, 15 marks)
%\begin{problem}
%{answer}
\item[(a)] \hspace{1cm}\\
\begin{align*}
\text{We have that: } n &= pq,\text{ } \phi(n) = \phi(p)\phi(q) = (p-1)(q-1)\\
\text{so, }\phi(n) &= pq + 1 - p - q\\
&= n + 1 - p - q \hspace{2cm} \textbf{(b/c pq = n)}\\
p + q &= n - \phi(n) + 1\\
q &=n - \phi(n) - p + 1\\
\text{since } n &= pq\\
n &= p(n - \phi(n) - p + 1)\\
&= p^2 - p(\phi(n) + 1)\\
\text{which means }& p^2 - p(\phi(n) + 1) -n = 0\\
\text{because we know n and }&\phi(n)\text{ we can solve this using the well known quadratic formula where,}\\
a &= 1\\
b &= (n - \phi(n) + 1)\\
c &= n\\
\text{making } p &= \frac{-b\pm \sqrt{b^2-4ac}}{2a}
\end{align*}
\item[(b)]\hspace{1cm}\\
\begin{doublespace}
We know that: $GCD(e_1,e_2) = 1$ so, $\exists$ $x,y$ such that $ x*e_1 + y*e_2 = 1 \pmod{n}$\\
Using the Extended Euclidean Algorithm, we can compute $x,y$\\
At which time we have $C_{1} * C_{2} = (M^{e_1})^{x} * (M^{e_2})^{y} = M^{e_1*x + e_2*y} = M^{1} = M$\\

\end{doublespace}
\item[(c)]\hspace{1cm}\\
Given that this message is encrypted for all of the key owners.\\
Also given that  $M < n_i $ for each $i$\\
Also given that  $1 \leq i \leq k$\\
According to the Chinese Remainder Theorem there is a unique $x < n_1 ∗
n_2 ∗ ... ∗ n_k$ such that for each $i$ we have $x = C_i \pmod n_i$
The CTR describes how we can compute x\\
$M^k < n_1∗n_2∗...∗n_k$, and also satisfies these equations. So, $x = Mk$
\item[(d)]\hspace{1cm}\\
%\end{answer}
%\end{problem}
\newpage
%\begin{problem}
%\begin{answer}
\pbitem (Fast RSA decryption using Chinese remaindering, 6 marks)
\begin{align*}
\textbf{Given that: }\\
C &\equiv M^e \pmod{n}\\
d_p &\equiv d( \pmod{p-1})\\
d_q &\equiv d(\pmod{p-1})\\
M_p &\equiv C^{d_p} \pmod{p}\\
M_q & \equiv C^{d_q} \pmod{q}\\
\textbf{We can say that: }\\
 C &\equiv M^e \pmod{p}\\
\textbf{because: }\\
C &\equiv M^e \pmod{n} \textbf{ so, }\\
C &= M^e + n(k) \textbf{ for some } k\\ 
&= M^e + p(qk) \textbf{ for some } k\\
\textbf{we can use similar logic to say that: }\\
C &\equiv M^e \pmod{q}\\
\textbf{so because, }\\
d_p &\equiv d \pmod{p-1}\\
 \textbf{we can say that: } d_p &= d + (p-1)(t) \textbf{ for some } t\\
 \textbf{so for, } M_p \pmod{p}\\
 M_p \equiv C^{d_p} \equiv C^{d + (p-1)(t)} &\equiv C^d * (C^{(p-1)})^{t} \equiv C^d \equiv M^{ed} \equiv M \pmod{p}\\
\textbf{ The same can be said for }&M_q \pmod{q}\\
\textbf{since } M_p &\equiv M \pmod{p} \textbf{ we can say that }\\
 M_p &= M + p(k) \textbf{ for some } k\\
 \textbf{and since } M_q &\equiv M \pmod{q} \textbf{ we can say that }\\
 M_q &= M + p(t) \textbf{ for some } t\\ 
\textbf{so, setting } M' &\equiv pxM_q + qyM_p \pmod{n}\\
\textbf{and subbing for }M_q \textbf{ and } M_p\\
M' &\equiv px(M +q(t)) + qy(M + p(k)) \pmod{pq}\\
M' &\equiv M(px) + \text{\sout{$pxq(t)$}} + M(qy) + \text{\sout{$pqy(k)$}} \pmod{pq}\\
M' &\equiv M(px + qy) \pmod{pq}\\
M' &\equiv M \pmod{pq}\qed\
\end{align*}
%\end{answer}
%\end{problem}
\newpage
\pbitem (An IND-CPA, but not IND-CCA secure version of RSA, 8 marks)\hspace{1cm}\\
\hspace{1cm}\\
\begin{align*}
\textbf{Given that: }\\
r &< n\\
s &= r^e \pmod{n}\\
t &= H(r) \oplus M\\
\textbf{and given that: }\\
\textbf{Encryption } \rightarrow C &= s||t = r^e \pmod{n} || H(r) \oplus M\\
\textbf{Decryption } \rightarrow M &= H( s^d \pmod{n}) \oplus t\\
&= H( r^{ed}\pmod{n}) \oplus H(r) \oplus M\\
&= H( r\pmod{n}) \oplus H(r) \oplus M\\
\textbf{Choosing a ciphertext of: }\\
C &= s||t \oplus M_1\\
\textbf{gives a decryption of: }\\
M_i &= H( r \pmod{n} ) \oplus H(r) \oplus M_i \oplus M_1\\
M_i &= H(r) \oplus H(r) \oplus M_i \oplus M_1\\
M_i &= M_i \oplus M_1\\
\textbf{Which means that we can tell with 100 } &\% \textbf{ certainty if the message is }\\ 
 M_1 \textbf{ or } M_2 \textbf{, i.e. If it decrypts to 0, then } &\textbf{ it is } M_1 \textbf{ otherwise it is } M_2\\
\end{align*}
\pbitem (Attacks on the ElGamal signature scheme, 23 marks)
%\begin{problem}
\item[(a)]
\item[(i)]\hspace{1cm}\\
\hspace{3cm}\bf{We are given:} $(r,s_1), (r,s_2)$\\
\hspace{3cm}\bf{assuming that intercepting }$(r,s_1), (r,s_2)$\\
\hspace{3cm}\bf{means we also have also intercepted }$M_1, M_2$ \bf{then, }\\
\begin{align*}
s_1 - s_2 &\equiv [H(M_1||r) - xr]K^{-1} - [H(M_2||r) - xr]K^{-1} \pmod{p-1}\\
&\equiv [H(M_1||r) - H(M_2||r)]K^{-1} \pmod{p-1}\\
\textbf{so if we knew } K\\
K (s_1 - s_2) &\equiv [H(M_1||r) - H(M_2||r)] \pmod{p-1}\\
\textbf{but we know that } &\gcd{((s_1-s_2),p-1)} = 1\\
\textbf{so we can find an inverse for }&(s_1 - s_2)\\ 
K (s_1 - s_2)(s_1 - s_2)^{-1} &\equiv [H(M_1||r) - H(M_2||r)](s_1 - s_2)^{-1} 
 \pmod{p-1}\\
K &\equiv [H(M_1||r) - H(M_2||r)](s_1 - s_2)^{-1} 
 \pmod{p-1}\\
\textbf{giving us } &K \textbf{ in terms of things we know, mainly, }\\
 &H(M_1||r)\textbf{, }H(M_2||r)\textbf{, }(s_1 - s_2)^{-1}  
\end{align*}
\item[(ii)]\hspace{1cm}\\
\begin{align*}
\textbf{Given that we know } &K \textbf{ and that we can re-arrange the equation}\\
xr & =H(M,r) -ks \pmod{p-1}\\
\textbf{Since we know } &r,k,s \textbf{ and }s H(M,r) \textbf{ and we know that } \gcd(r,(p-1)) = 1\\
\textbf{We can find }&\textbf{an inverse for } r \textbf{ giving, }\\
xrr^{-1} &\equiv r^{-1}(H(M,r) - ks \pmod{p-1})\\
x &\equiv r^{-1}(H(M,r) - ks \pmod{p-1})\\
\end{align*}
\item[(b)]\hspace{1cm}\\
\item[(i)]\hspace{1cm}\\
\begin{align*}
\textbf{We want to show that } v_1 = v_2 \textbf{ or }\\ y^{r} r^s &\equiv g^{M} \pmod{p}\\
y^{r} r^{-rv*} &\equiv g^{M}\pmod{p}\\
y^{r} (g^{u}y^{v})^{-rv*} &\equiv g^{M}\pmod{p}\\
y^{r}y^{vv*-r} g^{u-rv*} &\equiv g^{M}\pmod{p}\\
\text{\sout{$y^{r}y^{-r}$}} g^{u-rv*} &\equiv g^{M}\pmod{p}\\
g^{su} &\equiv g^{M}\pmod{p}\\
g^{M} &\equiv g^{M}\pmod{p}\\
\end{align*}
\item[(c)]\hspace{1cm}\\
\item[(i)]\hspace{1cm}\\
\begin{align*}
\textbf{Given }\\
R &\equiv rup-r(p-1) \pmod{p(p-1)}\textbf{ we can state that }\\
R &= rup -r(p-1) + p(p-1)(k)\text{ for some } k\\
&\textbf{taking the entire statement above } \pmod{p-1} \textbf{ we have}\\
R &\equiv rup -\text{\sout{$r(p-1) + p(p-1)(k)$}}\pmod{p-1}\\
&\textbf{We can re-write the equation as: }\\
R &\equiv ru(p-1) + ru \pmod{p-1}\textbf{ giving us: }\\
R &\equiv ru \pmod{p-1}\\
&\textbf{So, we can again re-write the equation as: }\\
R &= ru + (p-1)(t) \textbf{ for some t, and thus we can conclude that: }\\
y^R &\equiv y^{ru + t(p-1)} \equiv y^{ru} \text{\sout{$(y^{t})^{p-1}$}} \pmod{p}
\end{align*}
\item[(ii)]\hspace{1cm}\\
\begin{align*}
\textbf{Given }\\
R &\equiv rup-r(p-1) \pmod{p(p-1)}\textbf{ we can state that }\\
R &= rup -r(p-1) + p(p-1)(k)\text{ for some } k\\
&\textbf{taking the entire statement above } \pmod{p-1} \textbf{ we have}\\
R &\equiv \text{\sout{$rup$}} -r(p-1) + \text{\sout{$p(p-1)(k)$}}\pmod{p}\\
R &\equiv \text{\sout{$-rp$}} +r \pmod{p}\\
R &\equiv r \pmod{p}\\
&\textbf{ and given }\\
S &\equiv su \pmod{p-1}\textbf{ we can state that }\\  
S &= su + (p-1)(t) \textbf{ for some t}\\
&\textbf{ giving us }\\
R^s &\equiv R^{su + (k)(p-1)} \equiv R^{su} * \text{\sout{$(R^{k})^{(p-1)}$}} \pmod{p}\\
&\textbf{ and from the section above, }\\
R &\equiv r \pmod{p} \textbf{ so } R^{su} \equiv r^{su} \pmod{p}\\
\end{align*}
%\begin{answer}
%[Answer]
%\end{answer}
%\end{problem}
\end{problemlist}
\end{flushleft}
\end{document}